\section{Introduction}
\label{sec:chap7:1}
Les syst�mes interactifs (SI) � migrer peuvent avoir une d�composition fonctionnelle minimale de la figure %\ref{fig:chap3:1} 
qui comprend une interface utilisateur (UI) et un noyau fonctionnel (NF). La migration des SI vers la table interactive est un processus qui impacte � la fois le NF et l'UI. 


\subsection{Introduction}
La solution de migration assist\'{e}e que nous proposons est constitu\'{e}e 
de trois phase~: la phase d'extraction \`{a} partir de la structure 
analysable ({\it UIStruture}) d'une repr\'{e}sentation ({\it AUIStructure}) qui est d\'{e}crite dans le 
chapitre pr\'{e}c\'{e}dent, la phase d'adaptation de la structure de l'UI 
aux r\`{e}gles ergonomiques de la table interactive et enfin la phase de 
projection du mod\`{e}le d'UI adapt\'{e}, cette phase permet aussi aux 
concepteurs de personnaliser l'UI finale. 

\subsection{Adaptation de la structure de l'UI }
\label{subsec:adaptation}
Cette phase a pour finalit\'{e} la production d'une structure d'UI 
adapt\'{e}e aux r\`{e}gles ergonomiques de la table interactive. 
L'adaptation du mod\`{e}le de structure ({\it AUIStructure}) extraite de l'UI \`{a} migrer se 
fait en fonction des r\`{e}gles ergonomiques que nous pr\'{e}sentons \`{a} 
la section \ref{subsubsec:mylabel1}. La section 
\ref{subsubsec:identification} pr\'{e}sente une formalisation des r\`{e}gles 
d'adaptation de la structure et enfin la section 
\ref{subsubsec:mylabel2} est une synth\`{e}se du processus 
d'adaptation de la structure.

\subsubsection{R\`{e}gles ergonomiques pour l'adaptation du mod\`{e}le AUIStructure}
\label{subsubsec:mylabel1}
\paragraph{R\`{e}gles ergonomique pour l'adaptation des groupes d'interacteurs}
Les guidelines G1, G38, G39 et G40 sont rattach\'{e}es de G18 qui 
pr\'{e}conise de prendre en compte le nombre d'utilisateur et le partage de 
l'\'{e}cran pendant la mise en \oe uvre de l'UI. En effet ce principe impact 
la structuration des groupes d'interacteurs (G38, G38, G40) en fonction de 
leur contenu, elle impacte aussi l'accessibilit\'{e} de ces groupes qui 
doivent \^{e}tre 360\textdegree utilisable (G1). Les r\`{e}gles 
d'adaptations relatives \`{a} ces principes doivent consid\'{e}rer les types 
de Container ({\it Window, Simple, Panel }et{\it Table}). 

\paragraph{Guidelines d'utilisation des objets tangibles}
Les guidelines G15, G16 et G18 sont rattach\'{e}es \`{a} G5 pr\'{e}conise la 
consid\'{e}ration des objets physiques comme moyen d'interaction. Elles 
pr\'{e}cisent les cadres d'utilisation des objets tagu\'{e}s ou non. Les 
r\`{e}gles associ\'{e}es \`{a} ces guidelines ne sont pas appliqu\'{e}es 
dans tous les pendant cet \'{e}tape du processus, le concepteur doit 
pr\'{e}ciser dans s'il souhaite les appliquer avant la g\'{e}n\'{e}ration de 
l'UI propos\'{e}e.

\subsubsection{Identification des r\`{e}gles d'adaptation}
\label{subsubsec:identification}
Les r\`{e}gles identifi\'{e}es dans cette section permettent de 
g\'{e}n\'{e}rer la structure de l'UI pour la table interactive, cette 
structure est d\'{e}crite par l'ensemble $TargetUIStructure$ \`{a} l'aide 
des widgets de la plateforme cible.

\paragraph{R\`{e}gles d'adaptation des containers de type Table}
Les containers de type {\it Table} sont compos\'{e}s des {\it UIComponent} ayant des donn\'{e}es du 
m\^{e}me type (cf. chapitre 5.4.2.4), ils d\'{e}crivent un groupe 
d'interacteurs permettant de structurer des ressources de l'UI. Ce groupe se 
concr\'{e}tise en menu, ensemble de raccourcis ou en liste de donn\'{e}es de 
type (texte, image, etc.) sur une table interactive. La guideline G40 
pr\'{e}conise de garder le regroupement des donn\'{e}es structur\'{e}e pour 
facilit\'{e} son utilisation (la recherche d'information par exemple), en la 
combinant \`{a} la guideline G1 qui pr\'{e}conise l'utilisation des 
composants graphiques en 360\textdegree , nous d\'{e}duisons les r\`{e}gles 
d'adaptation 1 et 2 pour adapter les containers de type {\it Table} ayant des contenu 
que l'on souhaite rendre accessible \`{a} tous les utilisateurs de la table. 

La r\`{e}gle 1 concerne le cas o\`{u} les interacteurs fils d'un container 
de type {\it Table} contiennent des donn\'{e}es de type {\it Images, MediaElement ou Object} alors chaque interacteur fils 
doit \^{e}tre accessible \`{a} tout le monde (contenus dans un Widget ayant 
les primitives d'interactions intrins\`{e}ques {\bf Widget Move }et {\bf 
Widget Rotation).}

{\bf R\`{e}gle 1~:}$\, \forall \, container\, \in AUIStructure\wedge 
containerty??e=Table$
\[
\forall i\in containercontains\wedge \left( i.content.dataType\in \left\{ 
{\begin{array}{l}
 Image,\, \\ 
 MediaElement, \\ 
 Object \\ 
 \end{array}} \right\} \right)\wedge \, \exists \mathbf{w'}\in 
EquivalentWidget\left( i \right)\wedge \left\{ {\begin{array}{l}
 Widget\, Move, \\ 
 Widget\, Rotation\, \\ 
 \end{array}} \right\}\in \mathbf{w'}getIntrinsicInteraction()\, \, \wedge 
\, \exists \mathbf{w}\in EquivalentWidget\left( container \right)\wedge 
\left\{ {\begin{array}{l}
 Widget\, Move, \\ 
 Widget\, Rotation\, \\ 
 \end{array}} \right\}\notin \mathbf{w}getIntrinsicInteraction()\, 
\]
\[
\mathbf{w'}:est\, le\, widget\, \`{a}\, choisir\, pour\, concr\'{e}tiser\, 
les\, fils\, de\, container
\]
\[
\mathbf{w}:est\, le\, widget\, \`{a}\, choisir\, pour\, concr\'{e}tiser\, 
le\, container\, de\, type\, Table
\]
\[
container:{est\, l}^{'}interacteur\`{a}\, consid\'{e}rer\, pendant\, la\, 
g\'{e}n\'{e}ration\, de\, TargetAUIStructure
\]
La r\`{e}gle 2 permet d'adapter les containers de type {\it Table} contenant des 
interacteurs activable quelques soit le type de donn\'{e}es des interacteurs 
fils. Si tous les interacteurs fils ont les primitives 
d'interactions~effectives {\bf Widget Selection ou Navigation, Widget 
Display, Activation,} alors choisir un Widget qui permet de rendre le 
container accessible \`{a} tout le monde ({\bf Widget Move }et {\bf Widget 
Rotation).} 

{\bf R\`{e}gle 2~:} $\forall container\in AUIStructure\wedge 
containertype=Table$
\[
TRIAL RESTRICTION
\]
\[
TRIAL RESTRICTION
\]
\[
TRIAL RESTRICTION
\]
\[
TRIAL RESTRICTION
\]
\[
TRIAL RESTRICTION
\]
La r\`{e}gle 3 permet d'associer un objet physique \`{a} un container de 
type {\it Table}. Cette r\`{e}gle permet d'activer un menu par un objet ou un tag par 
exemple conform\'{e}ment aux guidelines G5, G15 et G16. Elle est optionnelle 
et d\'{e}pend du concepteur car tous les menus peuvent ne pas \^{e}tre 
activ\'{e}s par un objet physique.

{\bf R\`{e}gle 3~:} $TRIAL RESTRICTION$

Associer le Table \`{a} un objet physique pour l'afficher sur la table 
pendant la concr\'{e}tisation de {\it container}.

\paragraph{R\`{e}gles d'adaptation des containers de type Panel}
Les containers de type {\it Panel} repr\'{e}sentent un groupe d'interacteurs qui est 
compos\'{e} que des {\it UIComponent} avec des types de donn\'{e}es diff\'{e}rentes ou 
compos\'{e} \`{a} la fois de {\it Container} et d'{\it UIComponent}. Sur la table interactive, nous 
consid\'{e}rons qu'un {\it Panel} est groupe qui contient des interacteurs destin\'{e}s 
\`{a} un utilisateur et le concepteur peut d\'{e}cider de l'afficher par un 
objet physique par exemple. De mani\`{e}re concr\`{e}te un container de type 
Panel peut \^{e}tre un formulaire, une description d'un objet ou un ensemble 
de commandes pour acc\'{e}der \`{a} des fonctionnalit\'{e}s. Les Guidelines 
G1, G40 et G5 permettent d'\'{e}crire la r\`{e}gle 4 et la r\`{e}gle 5. La 
r\`{e}gle 5 est optionnelle et ne s'applique que lorsque le concepteur le 
souhaite.

{\bf R\`{e}gle 4~:} $TRIAL RESTRICTION$
\[
TRIAL RESTRICTION
\]
\[
TRIAL RESTRICTION
\]
\[
TRIAL RESTRICTION
\]
\[
TRIAL RESTRICTION
\]
\[
TRIAL RESTRICTION
\]
\[
TRIAL RESTRICTION
\]
{\bf R\`{e}gle 5~:}$TRIAL RESTRICTION$

Associer le panel \`{a} un objet physique pour l'afficher sur la table 
pendant la concr\'{e}tisation de {\it container}.

{\bf Exception~5:} Dans le cas o\`{u} un ou plusieurs {\it Panel} est fils d'un autre 
Panel, seuls le {\it Panel} le plus proche de la racine est consid\'{e}r\'{e} par les 
r\`{e}gles 4 et 5.

\paragraph{R\`{e}gle d'adaptation des containers de type Simple}
Les containers de type {\it Simple} ne contiennent que des containers, ils doivent 
\^{e}tre transform\'{e}s pour que chaque container puisse \^{e}tre 
utilis\'{e} par tous les utilisateurs suivant la guideline G1. 

Cependant dans le cas o\`{u} tous les containers fils est de type Simple, 
ils sont supprim\'{e}s et leurs fils sont consid\'{e}r\'{e}s comme fils du 
container courant, et r\'{e}cursivement. Cette exception permet de 
consid\'{e}rer que les groupes d'interacteurs pouvant \^{e}tre manipul\'{e}s 
par les utilisateurs de l'UI sur la table interactive.

{\bf R\`{e}gle 6~: }$TRIAL RESTRICTION$
\[
TRIAL RESTRICTION
\]
\[
TRIAL RESTRICTION
\]
\[
TRIAL RESTRICTION
\]
\[
TRIAL RESTRICTION
\]
{\bf Exception 6~: }$TRIAL RESTRICTION$

Si $TRIAL RESTRICTION$Alors remplacer les containers i par leur fils, si 
l'un des fils n'est pas un {\it Container} alors {\it container.type}$=${\it Panel} et appliquer la r\`{e}gle 4 
\'{e}ventuellement la r\`{e}gle 5 en fonction des pr\'{e}f\'{e}rences du 
concepteur.

La figure ci-dessous illustre le cas trait\'{e} par l'exception \`{a} la 
r\`{e}gle 6

\begin{figure}[htbp]
\centerline{\includegraphics[width=5.97in,height=2.23in]{Chapitre1.eps}}
\caption{Art\'{e}fact d'UI}
\label{fig1}
\end{figure}

\paragraph{R\`{e}gle d'adaptation des containers de type Window}
Les containers de type {\it Window} sont les racines du mod\`{e}le {\it AUIStructure}. La fen\^{e}tre 
principale d'une UI est adapt\'{e}e \`{a} la table interactive pour que les 
diff\'{e}rents \'{e}l\'{e}ments qui la compose puis \^{e}tre utilis\'{e}s en 
respectant G1. Tous fils imm\'{e}diats d'une fen\^{e}tre principale doivent 
\^{e}tre conformes \`{a} G1. 

Les autres fen\^{e}tres d'une UI (bo\^{\i}te de dialogue, formulaire, etc.) 
qui sont repr\'{e}sent\'{e}s par les containers de types {\it Window} sont 
transform\'{e}s en un groupe non dissociable mais conforme \`{a} G1.

La r\`{e}gle 7 s'applique pour les fen\^{e}tres principales des applications 
\`{a} migrer, elle permet rendre accessible tous les fils conforme \`{a} G1. 
Cette r\`{e}gle modifie le mod\`{e}le de structure de l'UI \`{a} migrer, le 
mod\`{e}le de structure d'UI pour la table surface doit avoir qu'un seul 
container de type {\it Window} apr\`{e}s l'application de cette r\`{e}gle. 

{\bf R\`{e}gle 7~}: 
\[
TRIAL RESTRICTION
\]
\[
TRIAL RESTRICTION
\]
\[
TRIAL RESTRICTION
\]
{\bf Exception~7: }$TRIAL RESTRICTION$ 
\[
TRIAL RESTRICTION
\]
\[
TRIAL RESTRICTION
\]
{\it container} est plac\'{e} comme fils de la fen\^{e}tre principale dans 
{\it TargetAUIStructure et }$TRIAL RESTRICTION$

La figure ci-dessous illustre un exemple de cas trait\'{e} par l'exception 
\`{a} la r\`{e}gle 7.

TRIAL RESTRICTION

\paragraph{Adaptation des interacteurs de type UIComponent }
Tous les interacteurs de type UIComponent non concern\'{e} par les 
r\`{e}gles ci-dessus sont r\'{e}utilis\'{e}s dans {\it TargetAUIStructure} et concr\'{e}tis\'{e}s en 
choisissant le {\it Widget} propos\'{e} par la fonction de classement des widgets 
\'{e}quivalents$TRIAL RESTRICTION${\bf .}

\paragraph{Adaptation d'AUIStructure en TargetAUIStructure}
Les r\`{e}gles ci-dessus modifient le mod\`{e}le de structure de l'UI de 
d\'{e}part en supprimant des interacteurs (cf. R\`{e}gle 6) ou en changeant 
la position des interacteurs dans l'arbre de structure (cf. R\`{e}gle 7). 
L'algorithme r\'{e}cursif $TRIAL RESTRICTION$ permet l'adaptation de chaque 
interacteur appartenant \`{a} {\it AUIStructure} pour g\'{e}n\'{e}rer le mod\`{e}le 
{\it TargetAUIStructure}. Cet algorithme utilise les R\`{e}gles 1, 2, 4, 6 et 7 pour adapter les 
containers. 

\underline {{\bf Algorithme}} $TRIAL RESTRICTION$\underline { }

\underline {{\bf Entr\'{e}e}}~:

\begin{itemize}
\item $TRIAL RESTRICTION$
\end{itemize}
\underline {{\bf Sortie}}~:

\begin{itemize}
\item $TRIAL RESTRICTION$
\end{itemize}
\underline {{\bf D\'{e}but}}

\underline {{\bf Si}} $TRIAL RESTRICTION$ \underline {{\bf Alors}}

\underline {{\bf Si}} $TRIAL RESTRICTION$ \underline {{\bf Alors}}

$TRIAL RESTRICTION
\quad
TRIAL RESTRICTION$ \underline {{\bf Alors}}
\[
TRIAL RESTRICTION
\]
\underline {{\bf Si}} $TRIAL RESTRICTION$ \underline {{\bf Alors}}
\[
TRIAL RESTRICTION
\]
\[
TRIAL RESTRICTION
\]
\underline {{\bf Sinon Si}} $TRIAL RESTRICTION$ \underline {{\bf Alors}}

\underline {{\bf Si}}$TRIAL RESTRICTION$ \underline {{\bf Alors}}
\[
TRIAL RESTRICTION
\]
\[
TRIAL RESTRICTION
\]
\underline {{\bf Sinon }}
\[
TRIAL RESTRICTION
\]
\[
TRIAL RESTRICTION
\]
\[
TRIAL RESTRICTION
\]
\underline {{\bf Sinon Si}} $TRIAL RESTRICTION$ \underline {{\bf 
Alors}}
\[
TRIAL RESTRICTION
\]
\underline {{\bf Return}} $TRIAL RESTRICTION$

\underline {{\bf Fin}}

Le mod\`{e}le {\it TargetAUIStructure }obtenu \`{a} l'issue de cette phase d'adaptation respect ont 
les propri\'{e}t\'{e}s suivantes~:

\begin{enumerate}
\item Il n'existe qu'un seul container de type {\it Window} appartenant \`{a} {\it TargetAUIStructure}
\item Tous les containers de type {\it Simple} sont des fils du container racine de type {\it Window} ou fils d'un container de type {\it Panel}.
\item Aucun container de type {\it Simple} n'est fils d'un container de type {\it Simple}
\end{enumerate}
\subsubsection{G\'{e}n\'{e}ration de l'UI propos\'{e}e}
\label{subsubsec:mylabel2}
L'UI propos\'{e}e est d\'{e}crite par sa structure {\it TargetUIStructure} et son comportement 
{\it TargetUIBehavior} qui sont g\'{e}n\'{e}rer \`{a} partir du mod\`{e}le de structure adapt\'{e} 
{\it TargetAUIStructure }et de {\it UIBehavior }(cf. Chapitre 4)$. $La concr\'{e}tisation du mod\`{e}le{\it TargetAUIStructure} en structure 
ex\'{e}cutable {\it TargetUIStructure} se fait en utilisant les r\`{e}gles d'adaptation ci-dessus. 
Les r\`{e}gles (R\`{e}gle 3 {\&} R\`{e}gle 5) permettant l'utilisation des 
objets tangibles comme moyen d'interactions sont appliqu\'{e}es pendant 
cette phase. En laissant le choix aux concepteurs d'appliquer ou non les 
r\`{e}gles 3 et 5, le processus de g\'{e}n\'{e}ration de {\it TargetUIStructure} parcours l'arbre 
{\it TargetAUIStructure } et pour chaque interacteurs retrouve les widgets \'{e}quivalents de la 
table interactive. 

La structure ex\'{e}cutable g\'{e}n\'{e}rer est constitu\'{e}e des widgets 
de la table interactive et elle est affich\'{e}e \`{a} l'\'{e}cran mais non 
utilisable car les composants graphiques ne sont pas plac\'{e}s suivant un 
layout pr\'{e}cis, les ressources de l'UI de ne sont pas d\'{e}finies, et 
elle n'est pas reli\'{e}e aux contr\^{o}leurs.

Les Handler du composant {\it TargetUIBehavior }sont g\'{e}n\'{e}rer \`{a} partir des primitives 
d'interactions effectives des interacteurs du mod\`{e}le de structure. 

Le composant {\it TargetUIBehaviorComponent }qui d\'{e}crit les m\'{e}thodes permettant la mise jour de la 
vue est adapt\'{e} aux widgets \'{e}quivalents choisis pendant la 
g\'{e}n\'{e}ration en modifiant automatiquement les m\'{e}thodes de type 
``{\it inputContainer}''.

\paragraph{Exemple d'applications de r\`{e}gles d'adaptation}
La figure ci-dessous pr\'{e}sente une illustration de l'application des 
r\`{e}gles d\'{e}finies \`{a} la section \ref{subsubsec:identification} pour 
la migration d'une UI Desktop vers une UI Table surface. Cette figure montre 
les types de containers de l'UI de d\'{e}part et les r\`{e}gles qui 
permettent leur adaptation sur la table interactive. En appliquant les 
r\`{e}gles 2 et 3 sur un container de type Table par exemple, on obtient un 
container de type Table et activable par un Tag (ou un objet physique). 

TRIAL RESTRICTION

\subsection{Projection de la structure de l'UI adapt\'{e}}
\subsubsection{S\'{e}lection des widgets}
La s\'{e}lection est un processus qui permet de retrouver l'ensemble des 
widgets \'{e}quivalents \`{a} un interacteur dans la biblioth\`{e}que 
graphique de la plateforme d'arriv\'{e}e. Ce processus se base sur des 
op\'{e}rateurs de comparaison qui permettent d'\'{e}tablir les 
\'{e}quivalences entre les interacteurs et les widgets. Cette section 
pr\'{e}sente les diff\'{e}rents op\'{e}rateurs de comparaison et 
l'algorithme de s\'{e}lection des widgets \'{e}quivalents.

\paragraph{Les op\'{e}rateurs de comparaison}
La comparaison entre les interacteurs et les widgets se basent sur les 
caract\'{e}ristiques qui leurs sont communes telles que les primitives 
d'interactions (effectives et intrins\`{e}ques), le type et la 
cardinalit\'{e} de donn\'{e}es. 

La comparaison des cardinalit\'{e} est effectu\'{e}e pour v\'{e}rifier si 
les widgets \'{e}quivalents supportent le m\^{e}me nombre de donn\'{e}es que 
l'interacteur \`{a} migrer. Elle est possible gr\^{a}ce \`{a} la fonction 
$TRIAL RESTRICTION$
\[
TRIAL RESTRICTION
\]
La prise en compte de la caract\'{e}ristique types de donn\'{e}es pour 
comparer les interacteurs et les widgets \`{a} pout but de v\'{e}rifier si 
l'adaptation des widgets \'{e}quivalents n\'{e}cessite l'ajout d'un 
adaptateur de types de donn\'{e}es. En effet si la cardinalit\'{e} et les 
primitives d'interaction d'un {\it Widget} correspondent \`{a} ceux d'un interacteur 
alors ce Widget sera propos\'{e} au concepteur comme \'{e}quivalent m\^{e}me 
si son utilisation n\'{e}cessite un travail suppl\'{e}mentaire (\'{e}criture 
du code pour adaptateur de type).~

La fonction $TRIAL RESTRICTION$ permet v\'{e}rifier l'\'{e}galit\'{e} des 
types de donn\'{e}es des interacteurs et des widgets. $TRIAL RESTRICTION$
\[
TRIAL RESTRICTION
\]
La prise en compte des primitives d'interaction par les op\'{e}rateurs 
implique la comparaison des sous-ensembles des primitives d'interactions qui 
permettent de caract\'{e}riser respectivement les interactions 
intrins\`{e}ques des widgets et les interactions effectives des 
interacteurs. Nous d\'{e}finissons trois op\'{e}rateurs de comparaisons de 
ces sous-ensembles de primitives d'interaction: 

\begin{itemize}
\item l'\'{e}galit\'{e} des deux sous ensemble: $TRIAL RESTRICTION$si toutes les primitives d'interaction effectives de l'interacteur se retrouvent dans le sous-ensemble des primitives effectives du Widget et si les deux sous ensembles de primitives d'interactions ont la m\^{e}me cardinalit\'{e}. Ceci est possible si une instance de {\it Widget} utilise toutes ses primitives d'interaction intrins\`{e}ques.
\item l'inclusion des primitives d'interaction effective d'un interacteur dans le sous-ensemble des primitives d'interaction d'un {\it Widget: }$TRIAL RESTRICTION$si toutes les primitives d'interaction du sous ensemble de gauche se retrouvent dans le sous-ensemble de droite. Ceci est possible si une instance de {\it Widget} n'utilise pas toutes ses primitives d'interaction intrins\`{e}ques.
\item l'inclusion des primitives d'interaction intrins\`{e}que d'un {\it Widget} dans le sous-ensemble des primitives d'interaction effective d'un interacteur$TRIAL RESTRICTION$. Ceci est possible si la biblioth\`{e}que graphique ne contient pas de widgets avec le sous ensemble de primitives d'interaction intrins\`{e}ques \'{e}quivalent \`{a} l'interacteur. Dans ces cas on consid\`{e}re les widgets qui n'impl\'{e}mentent pas les primitives d'interactions concernant les propri\'{e}t\'{e}s graphiques tout en conservant les autres primitives d'interactions. Les primitives d'interaction {\bf Widget Move}, {\bf Widget Rotation} et {\bf Widget Resize} sont les seules qui peuvent \^{e}tre omis par l'op\'{e}rateur de comparaison$TRIAL RESTRICTION$.
\end{itemize}
Les op\'{e}rateurs de comparaison combinent les trois caract\'{e}ristiques 
en consid\'{e}rant d'abord la cardinalit\'{e} des donn\'{e}es, car il n'y a 
pas d'\'{e}quivalence entre un interacteur et un Widget s'ils n'ont pas la 
m\^{e}me cardinalit\'{e}. Ensuite, les op\'{e}rateurs consid\`{e}rent les 
types des donn\'{e}es des interacteurs et enfin les sous-ensembles de 
primitives d'interaction. La combinaison des ces op\'{e}rateurs des 
caract\'{e}ristiques nous permet de relever quatre cas d'\'{e}quivalences 
signifiant pour le processus de migration~: le cas d'une \'{e}quivalence 
stricte (6.2.1.1) suivant les trois 
caract\'{e}ristiques ci-dessus, le cas d'une \'{e}quivalence large 
(6.2.1.2), le cas d'une \'{e}quivalence simple 
(6.2.1.3) et le cas d'une \'{e}quivalence faible 
(\ref{para:mylabel1}).

\begin{enumerate}
\item Equivalence stricte 
\end{enumerate}
Il ya une \'{e}quivalence stricte entre un interacteur et un Widget s'ils 
ont des cardinalit\'{e}s \'{e}gales, les m\^{e}me types de donn\'{e}es et 
des primitives d'interaction \'{e}gales. La fonction$TRIAL RESTRICTION $permet 
de v\'{e}rifier cette \'{e}quivalence, $TRIAL RESTRICTION$
\[
TRIAL RESTRICTION
\]
\begin{enumerate}
\item Equivalence large
\end{enumerate}
Il y a une \'{e}quivalence large entre un interacteur et un Widget s'ils ont 
des cardinalit\'{e}s \'{e}gales, les m\^{e}mes types de donn\'{e}es et si 
les primitives d'interaction de l'interacteur sont incluses dans celles du 
Widget. La fonction $TRIAL RESTRICTION$ permet de v\'{e}rifier cette 
\'{e}quivalence.$TRIAL RESTRICTION$
\[
TRIAL RESTRICTION
\]
\begin{enumerate}
\item Equivalence simple
\end{enumerate}
Il y a une \'{e}quivalence simple entre un interacteur et un Widget s'ils 
ont des cardinalit\'{e}s \'{e}gales, les types de donn\'{e}es 
diff\'{e}rentes et si les primitives d'interaction de l'interacteur sont 
\'{e}gales ou incluses dans celles du Widget. La fonction $TRIAL 
RESTRICTION$ permet de v\'{e}rifier cette \'{e}quivalence. $TRIAL 
RESTRICTION$
\[
TRIAL RESTRICTION
\]
\begin{enumerate}
\item Equivalence faible
\end{enumerate}
La fonction $TRIAL RESTRICTION$ permet de v\'{e}rifier cette 
\'{e}quivalence. $TRIAL RESTRICTION$
\[
TRIAL RESTRICTION
\]
\paragraph{Le processus de s\'{e}lection des widgets \'{e}quivalents}
\label{para:mylabel1}
L'algorithme utilis\'{e} par ce processus \`{a} pour objectif de retrouver 
les widgets \'{e}quivalents \`{a} chaque interacteur appartenant \`{a} 
{\it AUIStructure} en se basant sur les quatre op\'{e}rateurs d\'{e}crit ci-dessus. Les 
Widgets \'{e}quivalents sont plac\'{e}s dans le tableau de Widget 
\'{e}quivalent qui \`{a} quatre colonnes~repr\'{e}sentant les diff\'{e}rents 
op\'{e}rateurs utilis\'{e}s pour la recherche. La colonne {\bf strong} qui 
contient la liste des widgets strictement \'{e}quivalents, la colonne {\bf 
large} qui contient la liste des widgets d'\'{e}quivalence large, la colonne 
{\bf simple} qui contient la liste des widgets d'\'{e}quivalence simple et 
la colonne {\bf low }contient la liste des widgets d'\'{e}quivalence faible. 
Les widgets de ce tableau seront class\'{e}s en fonction des guidelines et 
du co\^{u}t qu'ils impliquent pour le concepteur en charge de migration. Le 
processus de s\'{e}lection parcourt pour chaque interacteur d'{\it AUIStructure} l'ensemble 
des {\it Widgets} d'une biblioth\`{e}que graphique.

\paragraph{Algorithme de s\'{e}lection}
\label{para:algorithme}
$TRIAL RESTRICTION:$ Biblioth\`{e}que graphique d'une table interactive

$TRIAL RESTRICTION$Le tableau de Widget \'{e}quivalent
\[
TRIAL RESTRICTION
\]
\[
TRIAL RESTRICTION
\]
Si $TRIAL RESTRICTION$ Alors
\[
TRIAL RESTRICTION
\]
Sinon Si $TRIAL RESTRICTION$ Alors
\[
TRIAL RESTRICTION
\]
Sinon Si $TRIAL RESTRICTION$ Alors
\[
TRIAL RESTRICTION
\]
Sinon Si $TRIAL RESTRICTION$ Alors
\[
TRIAL RESTRICTION
\]
\paragraph{Exemple de s\'{e}lection}
\label{para:exemple}
Consid\'{e}rons quelques widgets appartenant \`{a} cette 
interface~utilisateur de l'application d\'{e}crit dans le sc\'{e}nario du 
chapitre 1. La figure pr\'{e}sente~: un container d\'{e}crivant le menu 
principale ({\it File, Edition, etc.)}, un container contenant les widgets pour d\'{e}crire la liste 
d'image \`{a} utiliser, et une liste contenant les images utilisable pour la 
conception des bande dessin\'{e}es.

TRIAL RESTRICTION

Pour chaque interacteur correspondant aux Widgets d\'{e}crits ci-dessus, 
nous appliquons l'algorithme d\'{e}crit \`{a} la section 
\ref{para:algorithme} pour rechercher les widgets \'{e}quivalents 
dans la biblioth\`{e}que graphique XAML Surface (cf. annexe). Le tableau des 
widgets \'{e}quivalents obtenu correspond au tableau. Ce tableau nous montre 
que~:

\begin{table}[htbp]
\begin{center}
\caption{Equivalent Widget}
\begin{tabular}{|p{84pt}|p{66pt}|l|p{56pt}|l|}
\hline
{\bf Interactor Id/Name}& 
{\bf Strong}& 
{\bf Large}& 
{\bf Simple}& 
{\bf Low} \\
\hline
Id$=$5 \par Name$=$gridImage& 
Grid& 
ScatterViewItem& 
& 
 \\
\hline
Id$=$6 \par Name$=$listBox& 
& 
SurfaceListBox& 
LibraryContainer \par LibraryBar& 
 \\
\hline
Id$=$1 \par Name$=$mainMenu& 
Grid& 
ScatterViewItem& 
& 
 \\
\hline
Id$=$2 \par Name$=$menuItemFile& 
ElementMenu \par Grid& 
ScatterViewItem& 
LibraryContainer \par LibraryBar& 
 \\
\hline
Id$=$3 \par Name$=$menuItemOpen& 
ElementMenuItem \par Button& 
& 
Image& 
 \\
\hline
Id$=$4 \par Name$=$menuItemClose& 
ElementMenuItem \par Button& 
& 
Image& 
 \\
\hline
\end{tabular}
\label{tab1}
\end{center}
\end{table}

\begin{itemize}
\item le container {\it gridImage} est strictement \'{e}quivalent \`{a} {\it Grid}, le {\it Widget} {\it ScatterViewItem} a en plus les primitives d'interaction {\bf Widget Move} et {\bf Widget Resize}. L'op\'{e}rateur {\bf {\it simpleEquivalent}} ne trouve rien de plus que l'op\'{e}rateur {\bf {\it largeEquivalent}} car l'interacteur de d\'{e}part ({\it gridImage}) ne contient pas de donn\'{e}es~; par cons\'{e}quent il n'a pas de type de donn\'{e}e. L'op\'{e}rateur {\bf {\it lowEquivalent}} ne trouve rien car les autres colonnes du tableau ne sont pas vides. 
\item L'interacteur {\it listBox} n'a pas de Widget strictement \'{e}quivalent dans la biblioth\`{e}que XAML Surface car il n'impl\'{e}mente pas la primitive d'interaction {\bf Data Move In} qui est intrins\`{e}que \`{a} une {\it ListBox}. L'op\'{e}rateur {\bf {\it largeEquivalent}} retrouve {\it SurfaceListBox} car il d\'{e}crit des primitives d'interactions en plus. L'op\'{e}rateur simpleEquivalent permet de retrouver des widgets qui supportent d'autres types de donn\'{e}es. L'op\'{e}rateur {\bf {\it lowEquivalent}} ne trouve rien car les autres colonnes du tableau ne sont pas vides.
\item Le container contenant les \'{e}l\'{e}ments du menu principal ({\it mainMenu}) est strictement \'{e}quivalent \`{a} un {\it Grid}, il a une \'{e}quivalence large avec un {\it ScatterViewItem} (qui impl\'{e}mente les primitives d'interaction {\bf Widget Move} et {\bf Widget Resize} en plus), et n'as pas de widgets \'{e}quivalent simplement ou d'une \'{e}quivalence faible. Le choix du {\it ScatterViewItem} permettra d'avoir un menu d\'{e}pla\c{c}able et utilisable partout sur la table interactive. L'op\'{e}rateur {\bf {\it lowEquivalent}} ne trouve rien car les autres colonnes du tableau ne sont pas vides.
\item Les \'{e}l\'{e}ments du menu ({\it menuItemFile, menuItemEdit}) ont des widgets \'{e}quivalents suivants les op\'{e}rateurs d'\'{e}quivalence stricte, large ou simple. L'op\'{e}rateur {\bf {\it lowEquivalent}} ne trouve rien car les autres colonnes du tableau ne sont pas vides.
\end{itemize}

L'algorithme de s\'{e}lection permet d'avoir l'ensemble des widgets 
\'{e}quivalents pour un interacteur donn\'{e} dans le tableau des widgets 
\'{e}quivalents. L'utilisation des \'{e}l\'{e}ments de cet ensemble pour la 
migration est guid\'{e}e par les guidelines de la plateforme d'arriv\'{e}e 
qui permettra d'abord de classer les widgets \'{e}quivalents, ensuite de 
proposer une version d'UI migr\'{e}e et enfin pour faciliter la 
personnalisation de cette proposition.

\subsubsection{Classement des widgets}
\label{subsubsec:classement}
Le processus de classement des widgets \`{a} pour but d'assister les 
d\'{e}veloppeurs dans le choix des widgets de la plateforme d'arriv\'{e}e. 
Ce processus d\'{e}termine les `{\it meilleurs widgets'} de la classe d'\'{e}quivalence d'un 
interacteur \`{a} l'aide d'une fonction qui prend en compte deux 
crit\`{e}res~: les principes du guidelines et la charge de travail pour le 
programmeur. Cette fonction permet de consid\'{e}rer le classement des 
widgets \'{e}quivalents \`{a} un interacteur comme une instance d'un 
probl\`{e}me d'optimisation qui consiste \`{a} trouver les widgets qui sont 
conformes aux principes des guidelines (maximiser les crit\`{e}res conformes 
aux guidelines) et qui r\'{e}duit la charge de travail pour le 
d\'{e}veloppeur (minimiser la charge de travail). 

Cette section d\'{e}crit la prise en compte des guidelines 
(\ref{para:prise}) et la charge de travail pour le programmeur 
(\ref{subsubsec:charge}) comme crit\`{e}re pour le classement des 
widgets utilis\'{e}e par l'algorithme de classement 
(\ref{subsubsec:algorithme}).

\paragraph{Prise en compte des guidelines}
\label{para:prise}
La prise en compte des guidelines se fait en se basant sur les primitives 
d'interactions intrins\`{e}ques et les types de donn\'{e}es qui 
caract\'{e}risent les widgets de mani\`{e}re ind\'{e}pendante des 
biblioth\`{e}ques graphiques. Elle se fait d'abord en traduisant ses 
principes sous formes de r\`{e}gles appliqu\'{e}es sur les 
caract\'{e}ristiques des composants graphiques dans le but de les classer. 
Ensuite en calculant la conformit\'{e} de chaque widgets aux principes des 
guidelines formalis\'{e}es en r\`{e}gles.

\paragraph{Traduction des principes du guidelines en r\`{e}gles de classement des 
widgets \'{e}quivalents.}
Les principes des guidelines \`{a} consid\'{e}rer par le processus de 
classement sont celles qui facilitent le choix des widgets \'{e}quivalents. 
En consid\'{e}rant les 2 premiers principes d\'{e}crits par le chapitre 4~:

\begin{itemize}
\item {\bf G1}:{\it ``Provide a 360-Degree User Interface''}
\item {\bf G2}$:`` ${\it Make Experiences Natural and Better than Real}$''$ 
\end{itemize}
La traduction de ces deux principes pour le processus de classement des 
widgets \'{e}quivalents conform\'{e}ment aux deux \'{e}tapes 
pr\'{e}sent\'{e}s \`{a} la section (4.3.5) se fait en identifiant les 
\'{e}l\'{e}ments de l'UI qui permettront de faciliter le choix des widgets 
ensuite nous pr\'{e}ciseront comment ces \'{e}l\'{e}ments faciliterons le 
classement des widgets \'{e}quivalents.

{\bf G1} s'applique sur les primitives d'interactions car en 
privil\'{e}giant les widgets qui ont des primitives d'interactions 
interaction {\bf Widget Move }et {\bf Widget Rotation,} tous les 
utilisateurs auront aux widgets qui l'ont ind\'{e}pendamment de leur 
orientation ou de leur position autour de la table interactive. 

{\bf R\`{e}gle 1}: Les primitives d'interaction {\bf Widget Move }et {\bf 
Widget Rotation} sont privil\'{e}gi\'{e}s pour les containers.

{\bf G2} s'applique sur les types des donn\'{e}es car le choix des widgets 
ayant des types de donn\'{e}es images, son ou vid\'{e}o permet de 
r\'{e}v\'{e}ler leur contenu \`{a} l'utilisateur. 

{\bf R\`{e}gle 2}: Les types {\bf Image}, {\bf MediaElement et Object} sont 
mieux consid\'{e}r\'{e}s pour widgets ayant des contenus sur une table 
interactive.

Ces deux r\`{e}gles pr\'{e}cisent les \'{e}l\'{e}ments du mod\`{e}le des 
primitives d'interaction sur les quels peuvent s'appliquer les guidelines G1 
et G2 des guidelines de la table interactive. La conformit\'{e} par rapport 
\`{a} ces r\`{e}gles est d\'{e}termin\'{e}e par les fonctions d\'{e}crites 
ci-dessous.

\paragraph{Conformit\'{e} des widgets aux principes du guideline}
Pour marquer l'importance de certaines caract\'{e}ristiques (primitives 
d'interaction et type de donn\'{e}es) par rapport \`{a} d'autres, un poids 
est attribu\'{e} \`{a} celles qui sont conforme aux principes du guidelines. 
Le choix de la valeur du poids se fait par le programmeur avant la migration 
et ce choix permet de d\'{e}terminer s'il pr\'{e}f\`{e}re maximiser la 
conformit\'{e} aux principes du guidelines par rapport \`{a} la charge de 
travail ou s'il pr\'{e}f\`{e}re l'inverse. 

L'ensemble des poids des primitives d'interactions d'un Widget est 
d\'{e}termin\'{e} par la fonction$TRIAL RESTRICTION$ en utilisant le tableau 
ci-dessous.

L'ensemble des poids des types des donn\'{e}es d'un Widget est 
d\'{e}termin\'{e} par la fonction $TRIAL RESTRICTION$ en utilisant le 
tableau ci-dessous.

Ce tableau pr\'{e}cise un poids non nul pour les primitives d'interactions 
et les types de donn\'{e}es conformes aux R\`{e}gle 1 et 2 ci-dessus. 

Warning: TRIAL RESTRICTION -- Table omitted!

Nous formalisons le calcul de la conformit\'{e} d'un Widget aux principes 
des guidelines par la fonction$TRIAL RESTRICTION$. Cette fonction est la 
somme des fonctions repr\'{e}sentant les caract\'{e}ristiques suivantes 
prises en compte pour d\'{e}terminer la conformit\'{e} aux principes des 
guidelines~:

\begin{itemize}
\item la prise en compte des primitives d'interaction se fait par la fonction$TRIAL RESTRICTION$, elle permet de faire la somme des poids des primitives d'interaction intrins\`{e}ques qui sont conformes aux principes des guidelines. Les widgets \'{e}quivalents \`{a} un interacteur sont identifi\'{e}s par le processus dans la table {\bf {\it EquivalentWidget}}.
\end{itemize}
\[
TRIAL RESTRICTION
\]
\begin{itemize}
\item La prise en compte des types de donn\'{e}es est effectu\'{e}e par la fonction$TRIAL RESTRICTION$, cette fonction fait la somme des poids des types de donn\'{e}es des widgets \'{e}quivalents \`{a} un interacteur.
\end{itemize}
\[
TRIAL RESTRICTION
\]
La fonction $TRIAL RESTRICTION$ pour tout $TRIAL RESTRICTION${\bf 
}appartenant \`{a} la classe des widgets \'{e}quivalents \`{a} un 
interacteur.

\subsubsection{Charge de travail du programmeur}
\label{subsubsec:charge}
Le choix d'un Widget conforme aux principes des guidelines peut entrainer 
soit l'ajout d'un connecteur pour l'adaptation de type de donn\'{e}es, soit 
l'ajout des nouvelles donn\'{e}es pour l'interface utilisateur , soit 
l'ajout de codes suppl\'{e}mentaires pour la prise en compte des nouvelles 
primitives d'interactions. L'ajout d'un connecteur .....[Mehta, Medvidovic, 
and Phadke 2000] consiste \`{a} g\'{e}n\'{e}rer un code qui permet de faire 
un changement de type entre le type de donn\'{e}es de l'interacteur et le 
type de donn\'{e}es support\'{e} par les widgets \'{e}quivalents, cette 
t\^{a}che est automatisable pour l'ensemble des types de donn\'{e}es d'une 
biblioth\`{e}que graphique, elle n'entraine pas une charge de travail pour 
le programmeur.

Cependant, l'ajout des donn\'{e}es suppl\'{e}mentaires engendr\'{e} par le 
choix d'un Widget n'est pas automatisable car l'utilisateur doit trouver ou 
cr\'{e}er ces donn\'{e}es et ensuite faire le `{\it mapping'} avec les types de 
donn\'{e}es d\'{e}part. Par exemple le remplacement d'un menu classique dont 
les \'{e}tiquettes sont des chaines de caract\`{e}res avec un menu dont les 
\'{e}tiquettes sont des icones (Images) n\'{e}cessite la recherche ou la 
cr\'{e}ation de ces ic\^{o}nes. 

Par ailleurs, l'ajout de code suppl\'{e}mentaire pour la prise en compte des 
nouvelles primitives d'interaction est une t\^{a}che manuelle car les codes 
\`{a} ajouter d\'{e}pendent des primitives d'interactions intrins\`{e}ques 
\`{a} impl\'{e}ment\'{e}es.

\paragraph{Calcule de la charge de travail}
\label{para:calcule}
Les crit\`{e}res permettant le calcul de la charge de travail engendr\'{e}e 
par le choix d'un Widget sont~: la diff\'{e}rence entre le type de 
donn\'{e}es de l'interacteur et celui des widgets \'{e}quivalent, et les 
primitives d'interaction intrins\`{e}ques en plus des widgets 
\'{e}quivalents. A chacun de ces crit\`{e}res, le programmeur associe un 
co\^{u}t en se basant sur ses comp\'{e}tences pour r\'{e}aliser une 
t\^{a}che. L'estimation de ce co\^{u}t peut se faire en se basant sur la 
technique {\it Pomodoro} ....[Gobbo and Vaccari 2008] qui est utilis\'{e}e en {\it Extreme Programming} 
.................................[Beck 2000] pour permettre aux programmeurs 
d'optimiser leur temps de programmation en \'{e}valuant au mieux le temps de 
diff\'{e}rentes t\^{a}ches \`{a} effectuer.

La fonction$TRIAL RESTRICTION${\bf , }calcule la charge de travail 
engendr\'{e}e par le choix d'un widget \'{e}quivalent \`{a} un interacteur. 
C'est une somme des co\^{u}ts de chaque crit\`{e}re v\'{e}rifi\'{e} par le 
widget \`{a} choisir. Les crit\`{e}res sont~:

\begin{itemize}
\item la n\'{e}cessit\'{e} des donn\'{e}es suppl\'{e}mentaires \`{a} la suite d'un changement de type. Ce crit\`{e}re est v\'{e}rifi\'{e} si l'interacteur et le widget \`{a} choisir n'ont pas le m\^{e}me type de donn\'{e}es et si le type de donn\'{e}es du widget est {\bf Image}, {\bf MediaElement }ou {\bf Object.~}Le co\^{u}t de ce crit\`{e}re est donn\'{e} par la fonction$TRIAL RESTRICTION$ \`{a} l'aide du tableau ci-dessous qui permet aussi au programmeur de pr\'{e}ciser avant la migration le co\^{u}t de chaque op\'{e}tation.
\item L'utilisation des primitives d'interactions intrins\`{e}ques des widgets \'{e}quivalent. Ce crit\`{e}re est calcul\'{e} par la fonction$TRIAL RESTRICTION$ en utilisant le tableau ci-dessous. 
\end{itemize}
Dans ce tableau, le co\^{u}t de chaque t\^{a}che manuelle est estim\'{e} par 
le programmeur avant de commencer une migration. De nouvelles op\'{e}rations 
manuelles peuvent \^{e}tre d\'{e}finies, cette liste n'est pas exhaustive.

Warning: TRIAL RESTRICTION -- Table omitted!

La charge de travail est calcul\'{e}e par la fonction~:
\[
TRIAL RESTRICTION
\]
\subsubsection{Algorithme de classement }
\label{subsubsec:algorithme}
L'algorithme de classement d\'{e}termine le rang de chaque widget de la 
classe d'\'{e}quivalence d'un interacteur en faisant la diff\'{e}rence entre 
la conformit\'{e} aux guidelines et le charge de travail. La fonction$TRIAL 
RESTRICTION$d\'{e}termine la valeur de l'objectif d'un widget appartenant 
\`{a} une classe d'\'{e}quivalence. Cette valeur permet de classer le widget 
dans la classe d'\'{e}quivalence de l'interacteur.$TRIAL RESTRICTION$ 
\[
TRIAL RESTRICTION.
\]
Le `{\it meilleurs widget'} en consid\'{e}rant les principes du guidelines est celui qui a le max 
de la fonction$TRIAL RESTRICTION$
\[
TRIAL RESTRICTION.
\]
\paragraph{Exemple de classement}
Consid\'{e}rons les interacteurs {\it id}$=${\it 6} et {\it id}$=${\it 1} de l'exemple d\'{e}crit \`{a} la 
section \ref{para:exemple} et les widgets de leurs classes 
d'\'{e}quivalence. 

Consid\'{e}rons que le programmeur souhaite utilis\'{e}s les widgets les 
plus conformes aux principes des guidelines et pour cela fixe la valeur du 
poids des primitives d'interactions \`{a} P$=$2 et les co\^{u}ts C$_{i\, 
}=$1, 0\textless i\textless 6

Le widget {\it SurfaceListBox} de la classe d'\'{e}quivalence de l'interacteur id$=$6 a pour 
type de donn\'{e}es {\it String.} 
\[
TRIAL RESTRICTION
\]
\[
TRIAL RESTRICTION
\]
\[
TRIAL RESTRICTION
\]
Les widgets {\it LibraryContainer }et{\it LibraryBar} ont pour type de donn\'{e}es {\it Object}.
\[
TRIAL RESTRICTION
\]
\[
TRIAL RESTRICTION
\]
\[
TRIAL RESTRICTION
\]
Le widget ScatterView a en plus les primitives d'interactions Widget 
Rotation et Widget Move.
\[
TRIAL RESTRICTION
\]
\[
TRIAL RESTRICTION
\]
\[
TRIAL RESTRICTION
\]
Le widget Grid n'a pas de primitives d'interactions en plus ou des type de 
donn\'{e}es diff\'{e}rents
\[
TRIAL RESTRICTION
\]
\[
TRIAL RESTRICTION
\]
\[
TRIAL RESTRICTION
\]
Warning: TRIAL RESTRICTION -- Table omitted!

$^{\ast }${\it Les `meilleurs' widgets}

Le classement des widgets \'{e}quivalents facilite leur choix pour la 
prochaine \'{e}tape du processus de migration. Cette \'{e}tape consiste 
\`{a} proposer \`{a} aux concepteurs une premi\`{e}re version d'UI pour la 
table interactive. La mise en \oe uvre de cette premi\`{e}re version se fait 
utilisant le classement propos\'{e} \`{a} cette \'{e}tape et en 
consid\'{e}rant les principes des guidelines relatives \`{a} la structure de 
l'UI et aux interactions. 

\subsection{Personnalisation de l'UI propos\'{e}e}
\label{subsec:personnalisation}
La personnalisation de l'UI migr\'{e}e sur la table interactive \`{a} pour 
objectifs de produire une UI finale conforme aux attentes des concepteurs et 
qui respecte les crit\`{e}res ergonomiques li\'{e}s aux tables interactives. 
Cette \'{e}tape fait intervenir les concepteurs pour modifier et 
fa\c{c}onner la structure, les interactions et l'aspect visuel de l'UI 
g\'{e}n\'{e}rer automatiquement \`{a} la phase pr\'{e}c\'{e}dente. Le 
processus de personnalisation assiste les concepteurs pour chaque action de 
modification de l'UI propos\'{e}e. Cette section pr\'{e}sente les 
diff\'{e}rentes actions des concepteurs et l'assistance apport\'{e}e par le 
processus \`{a} l'aide des guidelines.

\subsubsection{Op\'{e}rations de modification de la structure}
La personnalisation de la structure d'UI propos\'{e}e permet aux concepteurs 
de remplacer, repositionner, redimensionner les composants graphiques$.$ Elle 
permet aussi d'effectuer les op\'{e}rations manuelles d\'{e}crite \`{a} la 
section \ref{para:calcule} telles que l'ajout de nouvelles 
ressources d'UI, l'ajout de code par exemple. Cette \'{e}tape permet enfin 
l'\'{e}dition des aspects visuels de l'UI (couleurs, taille, police, etc.) 
gr\^{a}ce \`{a} un \'{e}diteur graphique. Les op\'{e}rations manuelles de 
personnalisation de l'UI sont \'{e}valu\'{e}es par rapport \`{a} leurs 
conformit\'{e}s aux des guidelines de la table interactive. 

\paragraph{Remplacer des composants}
Cette op\'{e}ration permet de changer un ou plusieurs composants graphiques 
choisis pendant la phase pr\'{e}c\'{e}dente. Cette op\'{e}ration utilise les 
r\`{e}gles de classements des widgets \'{e}quivalents pour \'{e}valuer la 
conformit\'{e} aux guidelines (G1 et G2) et la charge de travail. 

\paragraph{Supprimer un composant graphique}
Cette op\'{e}ration permet de r\'{e}duire les composants graphiques dans le 
but de bloquer l'acc\`{e}s \`{a} des fonctionnalit\'{e}s non 
d\'{e}sir\'{e}es par l'utilisateur sur la table interactive. La guideline 
G19 justifie cette op\'{e}ration car elle pr\'{e}conise la r\'{e}duction des 
fonctionnalit\'{e}s~; le choix des fonctionnalit\'{e}s \`{a} conserver 
d\'{e}pend des applications \`{a} migrer et des objectifs de concepteur. Les 
composants graphiques supprim\'{e}s dans la structure {\it TargetUIStructure} sont aussi 
supprim\'{e}s dans le mod\`{e}le de structure {\it TargetAUIStructure.}

L'assistance permet \`{a} l'utilisateur de v\'{e}rifier si deux interacteurs 
activent la m\^{e}me m\'{e}thode du contr\^{o}leur. En effet les primitives 
d'interactions effectives du mod\`{e}le {\it TargetAUIStructure} permettent de retrouver l'ensemble 
des interacteurs qui permettent d'activer la m\^{e}me m\'{e}thode du 
Contr\^{o}leur. La $TRIAL RESTRICTION$ ci-dessous est utilis\'{e}e par 
les concepteurs avant la suppression d'un composant graphique pour retrouver 
les composants graphiques activants les m\^{e}mes fonctionnalit\'{e}s.
\[
TRIAL RESTRICTION
\]
\[
TRIAL RESTRICTION
\]
{\bf Si}$TRIAL RESTRICTION$ 

{\bf Alors}$TRIAL RESTRICTION$

\paragraph{Dupliquer un groupe de composants graphiques}
Cette op\'{e}ration \`{a} pour objectifs multiplier les composants 
graphiques pour permettre \`{a} tous les utilisateurs autour de la table d'y 
avoir acc\`{e}s facilement. Elle est pr\'{e}conis\'{e}e par les guidelines 
G7, G8 et G18. La duplication concerne les containers de type {\it Panel} ou {\it Table} et leurs 
fils car ils peuvent \^{e}tre concr\'{e}tis\'{e}s comme des menus, ou des 
groupes permettant la consultation des contenus. Les composants graphiques 
dupliqu\'{e}s dans la structure {\it TargetUIStructure} sont aussi dupliqu\'{e}s dans le mod\`{e}le 
de structure {\it TargetAUIStructure.}

L'assistance pour la duplication permet aux concepteurs de s\'{e}lectionner 
les interacteurs repr\'{e}sentant des menus ou des contenus \`{a} dupliquer.

\begin{itemize}
\item Les menus sont d\'{e}crits \`{a} l'aide des {\it Container} de type Table et qui contiennent des {\it UIComponent} ayant que les primitives d'interactions effectives de cet ensemble $TRIAL RESTRICTION$ 
\item Les groupes de composant permettant la consultation des contenus sont d\'{e}crits \`{a} l'aides des {\it Container} de type Table ou Panel qui contiennent des {\it UIComponent} avec des contenu ($TRIAL RESTRICTION)$ et n'ayant les primitives d'interaction effective$TRIAL RESTRICTION$.
\end{itemize}
\subsubsection{Op\'{e}rations de personnalisation des interactions}
Cette cat\'{e}gorie regroupe les op\'{e}rations qui permettent de 
d\'{e}finir l'utilisation des dispositifs d'interactions et les 
op\'{e}rations manuelles impos\'{e}es par le choix des widgets 
\'{e}quivalents. 

\paragraph{Associer un objet physique \`{a} un groupe de widgets }
Cette op\'{e}ration permet l'utilisation des objets tangibles comme moyens 
d'interaction pour afficher des composants graphiques ou pour activer. En 
effet les recommandations des guidelines G15 et G16 permet d'utiliser les 
objets physiques pour afficher un menu ou un formulaire. Elle concerne que 
les {\it Container} de type {\it Table} et de type {\it Panel.}

\paragraph{Compl\'{e}ter les codes }
Cette op\'{e}ration permet aux concepteurs de compl\'{e}ter les codes non 
g\'{e}n\'{e}rer automatiquement \`{a} la phase de g\'{e}n\'{e}ration de 
{\it TargetUIStructure}.

\subsubsection{Op\'{e}rations de personnalisation de l'aspect visuelle de l'UI}
L'ensemble des op\'{e}rations de personnalisation de l'aspect visuelle de 
l'UI pour permettre une utilisation naturelle de l'UI. 

\paragraph{D\'{e}finir un layout}
Cette op\'{e}ration permet de placer les \'{e}l\'{e}ments d'un {\it Container} de type 
{\it Panel} car ils sont empil\'{e}s ou dispos\'{e}s al\'{e}atoirement en fonction du 
l'impl\'{e}mentation du g\'{e}n\'{e}rateur de l'UI propos\'{e}e. Le 
concepteur doit d\'{e}finir le layout de chaque container de type {\it Panel} 
manuellement. Elle est conforme \`{a} la guideline G40 qui pr\'{e}conise de 
structurer les widgets d'un container qui est conforme \`{a} G1.

\paragraph{Ajouter les ressources de l'UI}
Cette op\'{e}ration permet aux concepteurs d'ajouter des ic\^{o}nes, des 
images ou d'autres ressources pour permettre une utilisation naturelle de 
l'UI, elle s'inspire de G4 qui pr\'{e}conise l'utilisation des composants 
graphiques ayant des contenus qui facilitent l'interaction.

\paragraph{D\'{e}finir le style de textes et les couleurs}
Ces op\'{e}rations permettent la personnalisation des textes descriptifs 
(\'{e}tiquettes, info-bulles, etc.). Elles sont d'\^{e}tre conformes aux 
guidelines G34, G35, G36, G45, G46, G47, G48, G50, G51.

\subsubsection{Synchronisation du mod\`{e}le TargetAUIStructure et TargetUIStructure}
Les op\'{e}rations d\'{e}crites ci-dessus sont effectu\'{e}es par le 
concepteur sur l'UI permettent d'abord de modifier {\it TargetUIStructure, }cependant les 
op\'{e}rations de personnalisation de la structure ont ensuite un impacte 
sur le mod\`{e}le de structure{\it TargetAUIStructure.} La synchronisation du mod\`{e}le et sa 
concr\'{e}tisation permettent de garder une coh\'{e}rence pour les 
op\'{e}rations de modifications suivantes.

\subsection{Synth\`{e}se }
Le processus de migration assist\'{e}e des UI pr\'{e}sent\'{e} dans ce 
chapitre et d\'{e}cri par la \tablename~\ref{tab1} est 
compos\'{e} de cinq \'{e}tapes dont une manuelle et quatre automatisables. 
Les \'{e}tapes automatisables sont l'extraction de la structure et des 
primitives d'interactions effectives, la s\'{e}lection (et le classement), 
la proposition de l'UI et la concr\'{e}tisation de l'application finale. 
L'\'{e}tape de personnalisation fait intervenir le concepteur.

La premi\`{e}re \'{e}tape du processus fait partie de la phase d'extraction 
la structure et des primitives d'interaction effective \`{a} partir de la 
structure analysable dans le but de d\'{e}crire l'UI ind\'{e}pendamment de 
la biblioth\`{e}que graphique. 

La deuxi\`{e}me \'{e}tape du processus est la s\'{e}lection des widgets 
\'{e}quivalents aux interacteurs de le la structure {\it AUIStructure} gr\^{a}ce aux 
op\'{e}rateurs de comparaison, et ensuite le classement des widgets 
\'{e}quivalents en prenant en compte les principes des guidelines. Cette 
\'{e}tape de la migration est automatique dans le but de facilit\'{e} le 
choix des widgets aux programmeurs en charge de la migration.

La troisi\`{e}me \'{e}tape du processus consiste \`{a} g\'{e}n\'{e}rer l'UI 
pour la plateforme d'arriv\'{e}e \`{a} l'aide des widgets \'{e}quivalents et 
en prenant en comptes les guidelines qui permettent l'adaptation de la 
structure, des interactions, du l'organisation des diff\'{e}rents widgets 
pour la plateforme cible.

La quatri\`{e}me \'{e}tape du processus fait intervenir le concepteur en 
charge de la migration en personnalisant l'UI propos\'{e}e par le processus 
de la phase pr\'{e}c\'{e}dente. Chaque op\'{e}ration de personnalisation de 
l'UI se fera en v\'{e}rifiant sa conformit\'{e} aux principes des 
guidelines. 

\begin{center}

TRIAL RESTRICTION

\end{center}

\subsubsection{Prise en compte des guidelines d'une table interactive }
Les \'{e}tapes du processus de migration vers la table interactive montre 
que les guidelines de la table interactive sont prise en compte pendant les 
\'{e}tapes de classement des widgets \'{e}quivalents, de proposition de l'UI 
et pendant la personnalisation de l'UI propos\'{e}e. Ces guidelines 
permettent de respecter les crit\`{e}res ergonomiques de conception 
.()..()()()(..)....()..()()()(..)........[Bastien and Scapin 1995] (R\'{e}f. 
Chapitre 3). La prise en compte de ces guidelines nous a permis de 
d\'{e}duire des r\`{e}gles de classement des widgets, des r\`{e}gles 
d'adaptation et des r\`{e}gles de g\'{e}n\'{e}ration de l'UI. 

Les r\`{e}gles de classement des widgets \'{e}quivalents permettent de 
pr\'{e}ciser les composants graphiques qui sont le plus conformes aux 
guidelines. En consid\'{e}rant le mod\`{e}le de guidelines propos\'{e}s par 
............()........[Vanderdonckt 1999], on note que les r\`{e}gles de 
classements sont conformes aux guidelines qui manipule les interacteurs ( 
repr\'{e}sent\'{e} par la Classe {\it AbstractInteractorObject}) qui ont les primitives d'interactions et 
les types de donn\'{e}es conseill\'{e}s.

Les r\`{e}gles d'adaptation de la structure et de g\'{e}n\'{e}ration de 
structure manipulent \`{a} la fois les interacteurs et dispositifs 
d'interactions conform\'{e}ment \`{a} ............()........[Vanderdonckt 
1999]. Les op\'{e}rations de personnalisations utilisent et manipulent aussi 
sur les interacteurs, les composants graphiques et les moyens 
d'interactions. Ceci nous montre que si les guidelines sont d\'{e}crites 
dans un mod\`{e}le proche de ............()........[Vanderdonckt 1999], 
elles peuvent \^{e}tre prise en compte dans un processus de migration d'UI 
vers une table interactive sous forme de r\`{e}gles et elles peuvent 
guid\'{e}es les op\'{e}rations du concepteur pendant la migration. 

Le mod\`{e}le de guidelines ci-dessous est celui que nous avons utilis\'{e} 
pour d\'{e}crire les guidelines de la table Microsoft Surface. 
