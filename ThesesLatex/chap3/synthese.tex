\subsubsection{Synthèse des approches de migrations spécifiques}Le processus de
migration ad-hoc de l'application AgilPlanner destinée à un desktop vers une
table interactive présente les points suivants par à nos problématiques: 

\begin{itemize}
\item la plateforme ciblée par le processus est une table interactive comme dans
notre cas. Les guidelines identifiées dans ce processus sont proches et
conformes des guidelines que nous avons décrites au chapitre~\ref{chap2}.
\item le processus de migration est certes manuel mais il est structuré en
plusieurs étapes. Les étapes de ce processus sont réutilisables pour d'autre
applications.
\end{itemize}
Les grandes lignes des guidelines identifiées par ce processus sont
réutilisables pour des applications similaires. Cependant dans certains cas les
guidelines doivent être plus formelles et basées sur des modèles abstraits de
l'UI pour être facilement réutilisables.

\paragraph{}Les processus de migration qui consiste à amener les applications
existantes sur des tables interactives sans concevoir à nouveau l'UI de départ
quand à lui est \textbf{réutilisable} et flexible. Cependant ce processus n'est
pas facile à mettre en place et ne permet pas la \textbf{prise en compte de
toutes guidelines} pour les table interactives.
\subsubsection{Synthèse des approches basées sur des modèles}
Ces approches ont pour but d'adapter des UI des applications sur différents
types plateformes en prenant en compte les spécificités des plateformes ciblées.
En s'appuyant sur des modèles d'UI, les approches décrivent des mécanismes
réutilisables pour d'autres applications. Ces approches sont aussi réutilisables
pour des tables interactives mais en adaptant les modèles et les
transformations. 
\paragraph{}La réutilisation des approches basées sur des modèles dans notre
contexte présente les limites suivantes:
\begin{itemize}
\item les modèles d'interactions abstraites permettant de décrire les activités
des UI ne sont pas exhaustives et doivent être multimodale
\item les correspondances entre le modèle CUI et les éléments des bibliothèques
graphiques sont faites de manières statiques à l'aide d'une table d'équivalence.
Ces correspondances sont limitées par ces tables d'équivalence dans le cas où
elles ne sont pas exhaustives.
\end{itemize}
