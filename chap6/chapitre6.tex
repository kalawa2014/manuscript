\section{Introduction}
\label{sec:chap6:1}

Le processus de migration que nous proposons est bas� sur les primitives d'interactions\footnote{Elles constituent aussi  un mod�le d'interactions abstraites} et un mod�le de structure de l'UI. Conform�ment aux approches bas�es sur les mod�les~\cite{Moore1997,Paterno'2009} �tudi�es au chapitre~\ref{chap3}, notre processus de migration comporte trois grande phases: \begin{itemize}
\item la premi�re consiste � d�crire l'application � migrer � l'aide des mod�les abstraits par un processus d'abstraction.
\item La deuxi�me phase consiste � transformer le mod�le de structure l'application de d�part pour la plateforme cible tout en prenant en compte les guidelines et en pr�servant au moins les interactions de d�part. Les interacteurs de la structure de l'UI transform�e sont des instances des composants graphiques de la plateforme cible. L'identification de ces interacteurs ce fait en \textbf{comparant les primitives d'interactions} effectives des interacteurs de l'UI source avec les primitives d'interactions des composants graphiques instanci�s de la plateforme cible.
\item La troisi�me consiste � produire l'UI finale de la plateforme cible en int�grant les aspects (style et layout)  qui ne sont pas pris en compte par les mod�les abstraits. La g�n�ration de l'UI finale consiste d'une part � s�lectionner pour chaque interacteur \textbf{le composant graphique �quivalent} conform�ment aux guidelines, d'autre part les composants s�lectionn�s sont positionn�s manuellement suivant les guidelines associ�es � la plateforme cible.
\end{itemize}

Les deux derni�re phase de ce processus de migration font appels � la \textbf{comparaison entre les interacteurs} en se basant sur les primitives d'interactions et la \textbf{s�lection de composant graphique} en fonction des primitives d'interactions. Ces deux notions d�finissent des �quivalences en se basant sur les primitives d'interactions entre les interacteurs des structures des UI sources et cibles d'une part et entre les interacteurs de l'UI cible et les composants graphiques de la plateforme cible d'autre part. 

Cette section se propose de d�finir et formaliser les op�rateurs de comparaison entre les interacteurs et les composants en se basant sur les primitives d'interactions.
